\documentclass[a4paper,10pt]{article}
\usepackage[english]{babel}
\usepackage{listings}
\usepackage{graphicx}

% Javascript definition
% Source: http://tex.stackexchange.com/questions/89574/language-option-supported-in-listings
\usepackage{color}
\definecolor{lightgray}{rgb}{.9,.9,.9}
\definecolor{darkgray}{rgb}{.4,.4,.4}
\definecolor{purple}{rgb}{0.65, 0.12, 0.82}
\lstdefinelanguage{JavaScript}{
  keywords={break, case, catch, continue, debugger, default, delete, do, else, false, finally, for, function, if, in, instanceof, new, null, return, switch, this, throw, true, try, typeof, var, void, while, with},
  morecomment=[l]{//},
  morecomment=[s]{/*}{*/},
  morestring=[b]',
  morestring=[b]",
  ndkeywords={class, export, boolean, throw, implements, import, this},
  keywordstyle=\color{blue}\bfseries,
  ndkeywordstyle=\color{darkgray}\bfseries,
  identifierstyle=\color{black},
  commentstyle=\color{purple}\ttfamily,
  stringstyle=\color{red}\ttfamily,
  sensitive=true
}

\lstset{
   language=JavaScript,
   backgroundcolor=\color{lightgray},
   extendedchars=true,
   basicstyle=\footnotesize\ttfamily,
   showstringspaces=false,
   showspaces=false,
   numbers=left,
   numberstyle=\footnotesize,
   numbersep=9pt,
   tabsize=2,
   breaklines=true,
   showtabs=false,
   captionpos=b
}
% end Javascript definition


\lstset {
	basicstyle=\footnotesize,
}
\title{Visualizing the World Wide Web with Visualtrace}


\author{\\Haji Akhundov\\h.akhundov@student.tudelft.nl\\ Delft University of Technology \and Koray Yanik\\k.i.m.yanik@student.tudelft.nl\\ Delft University of Technology}
        
\begin{document}
\maketitle

\begin{abstract}
To assist in network diagnostics the \emph{visualtrace} tool is designed to better visualize the results of the commonly used diagnostic tools \emph{traceroute} and \emph{tracert}.
\end{abstract}

\section{Introduction}
The internet is routed in a highly dynamic fashion. Several tools exist to gather diagnostic information about the chosen route from a certain machine to another, like the Unix tool \emph{traceroute}. The output of this tool is however very straightforward, textual, and requires some time to understand. This is even more the case when one wants to combine results to come to better conclusions. We propose a tool that shows this output in a more straightforward way and also combines multiple results in an intuitive way to provide better insight in networking issues and bottlenecks.

The tool consists of three parts:
\begin{itemize}
\item The parser is responsible for parsing output from a trace command into a graph structure;
\item The graph preprocessor merges multiple graphs into one;
\item The graph renderer visualizes the final graph.
\end{itemize}

\section{Parser}

\section{Graph preprocessor}

\section{Graph renderer}

\section{Conclusion}

\end{document}
